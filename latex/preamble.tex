\usepackage{amsmath}
\usepackage{amsthm}
\usepackage{amssymb}
\usepackage{graphicx}
\usepackage{multicol}
\usepackage{hyperref}
\usepackage{mathrsfs}
\usepackage{amsmath,amscd}
\usepackage[all,cmtip]{xy}
\usepackage{bbm}
\usepackage[utf8]{inputenc}
\usepackage[english]{babel}
\newcommand*{\titleGM}{\begingroup
	\hbox{
		\hspace*{0.2\textwidth}
		\rule{1pt}{\textheight}
		\hspace*{0.05\textwidth}
		\parbox[b]{0.75\textwidth}{
			
			{\noindent\LARGE\bfseries Applied Machine Learning and Predictive Modelling I: Modelling Stroke Data\\}\\[2\baselineskip]
			{\large \textbf{Authors:} Larissa Eisele, Fabian Lüthard, Yves Maillard} \\[0.5\baselineskip]
			{\large \textbf{Module:}   Applied Machine Learning and Predictive Modelling I} \\[4\baselineskip]
			{\large \textbf{Study Programm:}   Master of Science in Applied Information and Data Science} \\[4\baselineskip]
			{\large Submitted on 10th of June, 2022 } \\[1\baselineskip]
			{\large \textsc{ Supervisor: Matteo Tanadini and Daniel Meister }}
			
			\vspace{0.5\textheight}
			{\noindent Lucerne University of Applied Sciences and Arts }\\[\baselineskip]
	}}
	\endgroup}

\pagestyle{empty}
\makeatletter
\renewcommand{\maketitle}
{ \begingroup \vskip 10pt \begin{center} \Huge {\bf \@title}
		\vskip 10pt \large \@author \hskip 20pt \@date \end{center}
	\vskip 10pt \endgroup \setcounter{footnote}{0} }
\makeatother
\renewcommand{\labelenumi}{(\alph{enumi})}

\let\vaccent=\v
\renewcommand{\v}[1]{\ensuremath{\mathbf{#1}}}
\newcommand{\gv}[1]{\ensuremath{\mbox{\boldmath$ #1 $}}}

\newcommand{\uv}[1]{\ensuremath{\mathbf{\hat{#1}}}}
\newcommand{\abs}[1]{\left| #1 \right|}
\newcommand{\avg}[1]{\left< #1 \right>}
\let\underdot=\d
\renewcommand{\d}[2]{\frac{d #1}{d #2}}
\newcommand{\dd}[2]{\frac{d^2 #1}{d #2^2}}
\newcommand{\pd}[2]{\frac{\partial #1}{\partial #2}}

\newcommand{\pdd}[2]{\frac{\partial^2 #1}{\partial #2^2}}

\let\baraccent=\=
\renewcommand{\=}[1]{\stackrel{#1}{=}}
\providecommand{\wave}[1]{\v{\tilde{#1}}}
\providecommand{\fr}{\frac}
\providecommand{\RR}{\mathbb{R}}
\providecommand{\CC}{\mathbb{C}}
\providecommand{\NN}{\mathbb{N}}
\providecommand{\e}{\epsilon}
\newcount\colveccount
\newcommand*\colvec[1]{
	\global\colveccount#1
	\begin{pmatrix}
		\colvecnext
	}
	\def\colvecnext#1{
		#1
		\global\advance\colveccount-1
		\ifnum\colveccount>0
		\\
		\expandafter\colvecnext
		\else
	\end{pmatrix}
	\fi
}
\newtheorem{prop}{Proposition}
\newtheorem{thm}{Theorem}[section]
\newtheorem{problem}{Problem}[section]
\usepackage{cancel}
\newtheorem*{lem}{Lemma}
\theoremstyle{definition}
\newtheorem*{dfn}{Definition}
\theoremstyle{remark}
\newtheorem*{rmk}{Remark}
\newenvironment{solution}{
	\begin{trivlist} \item \textit{Solution}. }{
		\hspace*{\fill} $\blacksquare$\end{trivlist}}
\pagestyle{myheadings}